\documentclass[11pt,a4paper]{article}

% --- Pakker ---
\usepackage[utf8]{inputenc}
\usepackage[T1]{fontenc}
\usepackage[danish]{babel}
\usepackage{amsmath, amssymb}
\usepackage{siunitx}
\sisetup{locale=DE,detect-all}
\usepackage{graphicx}
\usepackage{microtype}
\usepackage{booktabs}
\usepackage{hyperref}
\usepackage{geometry}
\geometry{margin=2.5cm}
\usepackage{enumitem}

\date{}
\author{}

\begin{document}

\section*{Opgave 5: Elasticitet for potensfunktioner}
Et forskerhold har undersøgt forskellige mælkeproduktionsbedrifter i det sydlige Uruguay. De har set på hvordan forskellige græsningsbaserede driftsformer påvirker klimaaftrykket. Resultaterne viser, at store forskelle i bl.a.\ mælkeydelse pr.\ ko, fodersammensætning og arealanvendelse fører til betydelige variationer i klimaaftrykket. Hvis man vil gøre noget for klimaet, er det derfor vigtigt nøje at overveje mælkeproduktionens driftform (måden man laver mælkeproduktion på). 

I gennemsnit producerede gårdene \num{5672} kg mælk pr.\ ko pr.\ år, og spændet var fra ca.\ \num{3184} til \num{7772} kg. Samtidig lå klimaaftrykket fra \num{0.87} til \num{1.24} kg CO$_2$ pr.\ kg mælk med et gennemsnit på \num{0.99}.

En statistisk analyse viste, at klimaaftrykket havde en tydelig negativ sammenhæng med mælkeydelsen pr.\ ko: Jo mere mælk en ko producerer, desto mindre bliver klimaaftrykket pr.\ kg mælk. Denne sammenhæng kan beskrives med modellen:
\[
\mathrm{CF} \;=\; 16{,}367 \cdot \mathrm{MYC}^{-0{,}326},
\]
hvor \(\mathrm{MYC}\) er mælkeydelse (i kg pr.\ år pr. ko) og \(\mathrm{CF}\) er klimaaftrykket (i kg CO$_2$ pr.\ kg mælk)\footnote{Mere præcist måles MYC i kg fedt- og proteinkorrigeret mælk (FPCM) pr.\ år og CF (Carbon Footprint) i  CO$_2$-ækvivalenter pr. kg mælk.}. Der er tale om en såkaldt potensfunktion. Modellen viser en aftagende marginal effekt: Den største forbedring sker, når mælkeydelsen stiger fra lave til moderate niveauer, mens gevinsten flader ud ved meget høje ydelser.\footnote{Data stammer fra: 
\url{https://www.researchgate.net/publication/269989556_Practices_to_Reduce_Milk_Carbon_Footprint_on_Grazing_Dairy_Farms_in_Southern_Uruguay_Case_Studies} (Practices to Reduce Milk Carbon Footprint on Grazing Dairy Farms in Southern Uruguay: Case Studies)}.

\section*{Elasticitet}
Den omtalte model er en potensfunktion
\begin{equation*}
f(x) \;=\; b \cdot x^{a},
\end{equation*}
hvor $a$ kaldes \emph{eksponenten} og $b>0$ er en proportionalitetskonstant. For sådanne modeller gælder den præcise procentsammenhæng.
\begin{equation*}
1 + r_y \;=\; (1 + r_x)^{a},
\end{equation*}
når $x$ ændres med $r_x$ (f.eks.\ $r_x=0{,}10$ for $10\%$), og funktionsværdien $y=f(x)$ dermed ændres med $r_y$.

\:

\textbf{Hvordan læses formlen?} Hvis $x$ ganges med $(1+r_x)$, så ganges $f(x)$ med $(1+r_y)=(1+r_x)^a$. Er $a>0$, stiger $y$ når $x$ stiger; er $a<0$, falder $y$ når $x$ stiger. Når $|a|<1$, er effekten \emph{elastisk dæmpet}: en given procentvis ændring i $x$ giver en mindre procentvis ændring i $y$ (men modsat rettet, hvis $a<0$).

\:

\textbf{Elasticitet.} I en potensfunktion er $a$ lig med den (konstante) elasticitet: for små ændringer gælder den lineære tilnærmelse
\begin{equation*}
r_y \approx a \cdot r_x,
\end{equation*}
mens den eksakte sammenhæng altid er $1+r_y=(1+r_x)^a$. Tilnærmelsen er god for små procenter (fx $\pm 1$--$5\%$) og bliver gradvist mindre præcis for større ændringer.

Elasticiteten beskriver, hvor mange procent $y$ ændres, når $x$ ændres med én procent. For en potensfunktion $f(x)=b\cdot x^a$ bestemmes elasticiteten ved:
\[
\frac{f'(x) \cdot x}{f(x)} = \frac{a \cdot b \cdot x^{a-1} \cdot x}{b \cdot x^a} = a.
\]
Det viser, at elasticiteten er konstant og netop lig eksponenten $a$. Det betyder, at en $1\%$ ændring i $x$ giver omtrent en $a\%$ ændring i $y$. Hvis $a=-0{,}326$ som i vores model, betyder det for eksempel, at en $1\%$ stigning i $x$ (mælkeydelsen) medfører et fald i $y$ (klimaaftrykket) på $0{,}326\%$. For små ændringer kan man derfor bruge den lineære tilnærmelse:
\[
r_y \approx a \cdot r_x,
\]
som er en førsteordens-approksimation af den eksakte sammenhæng $1+r_y=(1+r_x)^a$ (dvs. $r_y = a \cdot r_x$ er ligningen for tangenten til grafen for $r_y=(1+r_x)^a-1$ i punktet $(0,0)$.


\:

\textbf{Anvendt på modellen her.} I denne opgave er $a=-0{,}326<0$, så højere MYC medfører lavere CF, og effekten er dæmpet i størrelsesordenen $|a|=0{,}326$.

\section*{Opgave}
Vi betragter modellen $\mathrm{CF}=16{.}367\cdot \mathrm{MYC}^{-0{,}326}$. Brug procentsammenhængene:
\[
1+r_{\mathrm{CF}}=(1+r_{\mathrm{MYC}})^{-0{,}326} \text{    og    } r_{\mathrm{CF}} \approx a \cdot r_{\mathrm{MYC}}
\]
til at besvare følgende (svar med \% afrundet passende).

\begin{enumerate}[label=\alph*)]
    \item \textbf{Graf.} Tegn først grafen for potensfunktionen med MYC mellem fra 2000 til 10000
    
        \item \textbf{Fortolkning af sammenhæng.}  
        En mælkeproducent øger mælkeydelsen pr.\ ko fra \SI{4000}{kg} til \SI{6000}{kg} pr.\ år.  
        Brug modellen til at bestemme klimaaftrykket pr.\ kg mælk før og efter ændringen, og beregn den procentvise forskel.  
        Kommentér kort, hvad dette siger om potentialet for at reducere CO$_2$-udledningen gennem øget mælkeydelse.
    
    \item \textbf{Elasticitet og tilnærmelse.} Brug $a=-0{,}326$ til at:
        \begin{enumerate}[label=\arabic*)]
        \item forklare kort, hvad tallet $-0{,}326$ betyder i ord,
        \item estimere ændringen i CF ved en $10\%$ stigning i MYC med den lineære regel $r_y\approx a r_x$,
        \item sammenligne dette med den \emph{eksakte} ændring via $1+r_{\mathrm{CF}}=(1+r_{\mathrm{MYC}})^{-0{,}326}$ og at kommentere forskellen.
        \end{enumerate}
    
    \item \textbf{Formlen for elasticitet: $E=\frac{f'(x) \cdot x}{f(x)}$}
        
        Elasticiteten er 'forholdet mellem den procentvise ændring i $y$ og den procentvise ændring af $x$'. Den procentvise ændring i $y$ kan skrives som $\frac{\Delta y}{y}$. Den procentvise ændring i $x$ kan skrives som $\frac{\Delta x}{x}$. Forholdet mellem $\frac{\Delta y}{y}$ og $\frac{\Delta x}{x}$ er så: 
        $
        \frac{\frac{\Delta y}{y}}{\frac{\Delta x}{x}}
        $
        \begin{enumerate}[label=\arabic*)]
        \item Vis at
        \[
        \lim_{\Delta x \to \infty}
        \frac{\frac{\Delta y}{y}}{\frac{\Delta x}{x}}=\frac{f'(x) \cdot x}{f(x)}
        \]
        ved bl.a. at bruge brøkregneregler på $\frac{\frac{\Delta y}{y}}{\frac{\Delta x}{x}}$
        \end{enumerate}
    
    \end{enumerate}

\newpage
\section*{Løsninger}

\begin{enumerate}[label=\alph*)]
    \item ...
    
    \item \textbf{Fortolkning af sammenhæng.}
    Vi bruger modellen $\mathrm{CF}=16{.}367\cdot \mathrm{MYC}^{-0{,}326}$.
    
    \textit{Før (\textup{MYC} $=\SI{4000}{kg}$):}
    
    \[
    \mathrm{CF}(4000)=16{.}367\cdot 4000^{-0{,}326}
       \approx \SI{1.096}{kg\ CO_2/kg\ \text{mælk}}.
    \]
    
    \textit{Efter (\textup{MYC} $=\SI{6000}{kg}$):}
    \[
    \mathrm{CF}(6000)=16{.}367\cdot 6000^{-0{,}326}\approx \SI{0.960}{kg\ CO_2/kg\ \text{mælk}}.
    \]
    
    \textit{Procentvis ændring:}
    \[
    \frac{0.960-1.096}{1.096}\cdot 100\% \;\approx\; -\SI{12.38}{\percent}.
    \]
    Alternativt via procentsammenhængen:
    \[
    r_{\mathrm{MYC}}=\frac{6000}{4000}-1=0.5,
    \qquad
    r_{\mathrm{CF}}=(1+0.5)^{-0.326}-1\approx -0.1238 = -\SI{12.38}{\percent}.
    \]
    
    \textit{Kommentar.} En stigning i ydelsen fra 4000 til 6000 kg pr.\ ko reducerer klimaaftrykket pr.\ kg mælk med ca.\ \SI{12.4}{\percent}. Det afspejler, at eksponenten $a=-0{,}326<0$ (negativ elasticitet): højere ydelse giver lavere CF pr.\ kg. Effekten har dog aftagende marginale gevinster, og den samlede udledning pr.\ gård afhænger også af produktionens størrelse, fodringsstrategi m.m.
    
    \item (1) Elasticiteten er konstant og lig $a=-0{,}326$: en lille ændring i MYC på $1\%$ giver omtrent en ændring i CF på $-0{,}326\%$.\\
    (2) Lineær tilnærmelse ved $10\%$: $r_y\approx a \cdot r_x=-0{,}326\cdot 0{,}10=-0{,}0326$ dvs. \SI{-3.26}{\percent}.\\
    (3) Med $r_{MYC}=0{,}10$ fås
    $r_{\mathrm{CF}}\approx (1+0{,}10)^{-0{,}326}-1 \approx -0{,}03059 \;=\; \SI{-3.06}{\percent}$
    Eksakt værdi ved $10\%$ er \SI{-3.06}{\percent}, så tilnærmelsen \emph{overvurderer} faldet med ca.\ $0{,}20$ procentpoint.
    
    \item For grænseværdien skal vi vise, at der gælder:
    \[
    \lim_{\Delta x \to 0} \frac{\frac{\Delta y}{y}}{\frac{\Delta x}{x}}
    = \frac{f'(x)\ \cdot x}{f(x)}.
    \]
    Vi kan omskrive $\frac{\frac{\Delta y}{y}}{\frac{\Delta x}{x}}=\frac{\Delta y}{\Delta x}\cdot \frac{x}{y}$. Vi bemærker, at pr. definition har vi $\lim_{\Delta x \to 0} \frac{\Delta y}{\Delta x}=f'(x)$. Det viser, at 
    
    \[
    \lim_{\Delta x \to 0} \frac{\frac{\Delta y}{y}}{\frac{\Delta x}{x}}
    =\lim_{\Delta x \to 0} \frac{\Delta y}{\Delta x}\cdot \frac{x}{y}= \frac{f'(x) \cdot x}{f(x)}.
    \]

\end{enumerate}

\end{document}
