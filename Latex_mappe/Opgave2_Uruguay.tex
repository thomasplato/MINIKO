
\documentclass[11pt,a4paper]{article}

% --- Pakker ---
\usepackage[utf8]{inputenc}
\usepackage[T1]{fontenc}
\usepackage[danish]{babel}
\usepackage{amsmath, amssymb}
\usepackage{siunitx}
\sisetup{locale=DE,detect-all}
\usepackage{graphicx}
\usepackage{microtype}
\usepackage{hyperref}
\usepackage{geometry}
\geometry{margin=2.5cm}
\usepackage{enumitem}

\author{}
\date{}

\begin{document}

\section*{Opgave 2: Mælkeproduktion pr.\ hektar og klimaaftryk}
Et forskerhold har undersøgt forskellige mælkeproduktionsbedrifter i det sydlige Uruguay. I en forskningsartikel analyseres 24 græsningsbaserede mælkeproduktionsbedrifter med fokus på klimaaftrykket, CF (Carbon Footprint). 

Et centralt fund er en negativ sammenhæng mellem mælkeproduktion pr.\ hektar pr.\ år (\(\mathrm{MPA}\), målt i kg/ha/år) og klimaaftrykket (\(\mathrm{CF}\) målt som CO$_2$ pr. kg mælk)\footnote{Mere præcist måles MPA i kg fedt- og proteinkorrigeret mælk (FPCM) pr.\ ha pr.\ år og CF i  CO$_2$-ækvivalenter pr. kg mælk}. Helt overordnet: når flere kg mælk produceres pr.\ hektar pr. år, falder klimaaftrykket pr.\ kg mælk. Sammenhængen kan beskrives ved en potensmodel:
\[
\mathrm{CF} \;=\; 6{.}024 \cdot \mathrm{MPA}^{-0{,}219}.
\]
Modellen viser, at CF falder, men i aftagende takt, når MPA øges. Artiklen opdeler også bedrifterne i tre klynger med lav, mellem og høj produktion; de højtydende bedrifter har lavere CF, men ofte højere kvælstofbelastning pr.\ hektar. Dermed peger resultaterne både på effektivitetsgevinster og på mulige miljømæssige trade-offs.\footnote{Data stammer fra: 
\url{https://www.researchgate.net/publication/269989556_Practices_to_Reduce_Milk_Carbon_Footprint_on_Grazing_Dairy_Farms_in_Southern_Uruguay_Case_Studies} (Practices to Reduce Milk Carbon Footprint on Grazing Dairy Farms in Southern Uruguay: Case Studies)}.

\section*{Graf}
\begin{center}
\includegraphics[width=0.7\linewidth]{graph_mpa_cf.pdf}\\
\small (Markerede punkter: MPA = 2000, 4000, 6000, 8000 kg/ha/år. Stiplet: mål CF = 0.90 og tilsvarende MPA.)
\end{center}

\section*{Opgave}
Vi betragter modellen
\[
\mathrm{CF} \;=\; 6{.}024 \cdot \mathrm{MPA}^{-0{,}219},
\]
hvor \(\mathrm{CF}\) er klimaaftrykket målt i kg CO$_2$ pr.\ kg mælk og \(\mathrm{MPA}\) er mælkeproduktion målt i kg mælk/ha/år.

\begin{enumerate}[label=\alph*)]
    \item \textbf{Beregninger.} Bestem CF for \(\mathrm{MPA} = \num{2000}, \num{4000}, \num{6000}\) og \(\num{8000}\) kg/ha/år.
    \item \textbf{Fordoblingseffekt.} Hvor stor en procentvis ændring i CF opnås ved at fordoble MPA (1) fra 2000 til 4000 og (2) fra 4000 til 8000? Kommentér dine resultater.
    \item \textbf{Elasticitet.} En potensmodel \(\mathrm{CF} = b \cdot \mathrm{MPA}^{a}\) har en \emph{elasticitet} \(a\), dvs.\ en 1\% stigning i MPA ændrer CF med ca.\ \(a\)\%. Bestem elasticiteten i vores model og fortolk tallet.
    \item \textbf{Målsætning.} Hvilken MPA kræves (mindst) for at nå \(\mathrm{CF} \le \num{0.90}\)? Tjek at din løsning passer med figuren ovenover. 
    \item \textbf{Diskussion.} Hvilke mulige trade-offs kan opstå, når man øger MPA for at reducere CF pr.\ kg mælk? Inddrag fx kvælstof pr.\ hektar, foderstrategi eller dyrevelfærd (denne diskussion rækker ud over matematikken).
\end{enumerate}

\newpage
\section*{Løsninger}
\begin{enumerate}[label=\alph*)]
    \item \textbf{CF-beregninger.}
    \[
    \textup{For eksempel for MPA} = \SI{2000}{}:\\[4pt]
    \mathrm{CF}(\num{2000}) = \num{6.024} \cdot \num{2000}^{\num{-0.219}}
       = \SI{1.140}{\kilogram\ CO_2\per\kilogram\ \text{mælk}},\\[6pt]
    \]
    \[
    \begin{aligned}
    \mathrm{MPA}=2000:&\quad \mathrm{CF} = \num{1.140},\\
    \mathrm{MPA}=4000:&\quad \mathrm{CF} = \num{0.980},\\
    \mathrm{MPA}=6000:&\quad \mathrm{CF} = \num{0.896},\\
    \mathrm{MPA}=8000:&\quad \mathrm{CF} = \num{0.842}.
    \end{aligned}
    \]
    
    \item \textbf{Fordoblingseffekt.} Fra forrige opgave ved vi, at CF$_{\mathrm{MPA}=2000}= \num{1,140}$ og CF$_{\mathrm{MPA}=4000}= \num{0.980}$. Det procentvise fald, når MPA øges fra 2000 til 4000, er da:
    
    \[
    \frac{\mathrm{CF}_{\mathrm{MPA}=4000}}{\mathrm{CF}_{\mathrm{MPA}=2000}}-1=\frac{\num{0.980}}{\num{1.140}}-1=\num{-0.14}=\num{-14}\%
    \]
    
    Tilsvarende får vi 14$\%$ fald med CF$_{\mathrm{MPA}=4000}= \num{0,980}$ og CF$_{\mathrm{MPA}=8000}= \num{0,842}$. Man kan vise, at ved en fordobling af MPA ændres CF med faktoren \(2^{-0{,}219}\), dvs.\ en procentvis ændring på
    \[
    \bigl(2^{-0{,}219}-1\bigr))\cdot 100\% \approx \num{-14}\%.
    \]
   En potensmodel giver den samme procentvise effekt pr.\ fordobling - lige meget om MPA går fra 2000 til 4000 eller fra 4000 til 8000.    
   
    \item \textbf{Elasticitet.} \(a=-0{,}219\). En 1\% stigning i MPA reducerer CF med ca.\ \(\num{0.219}\%\) (negativ sammenhæng).
    
   
    \item \textbf{Målsætning.} Vi løser ligningen \(0{,}90 = 6{.}024 \cdot \mathrm{MPA}^{-0{,}219}\):
    \[
    \mathrm{MPA} = \left(\frac{0{,}90}{6{.}024}\right)^{1/-0{,}219} = \num{5889} \text{ kg/ha/år}.
    \]
    
    \item \textbf{Diskussion.} Øget MPA kan kræve mere intensiveret drift og/eller højere fodereffektivitet. Artiklen indikerer, at højere MPA hænger sammen med lavere CF pr.\ kg mælk, men at kvælstofbelastning pr.\ hektar kan stige i de mest intensive systemer. Derfor skal reduktion i CF afvejes mod lokale miljøpåvirkninger og driftsmæssige hensyn.
\end{enumerate}

\end{document}
